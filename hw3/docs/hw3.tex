\documentclass[journal, letterpaper, draftclsnofoot, onecolumn, 10pt]{IEEEtran}

\usepackage{graphicx}
\usepackage{amssymb}
\usepackage{amsmath}
\usepackage{amsthm}

\usepackage{alltt}
\usepackage{float}
\usepackage{color}
\usepackage{url}
\usepackage{listings}

\usepackage{balance}
\usepackage[TABBOTCAP, tight]{subfigure}
\usepackage{enumitem}
\usepackage{pstricks, pst-node}
\usepackage{placeins}
\usepackage{geometry}
\usepackage{hyperref}
\geometry{textheight=8.5in, textwidth=6in}

\lstset{
  language=C,                % choose the language of the code
  numbers=left,                   % where to put the line-numbers
  stepnumber=1,                   % the step between two line-numbers.
  numbersep=5pt,                  % how far the line-numbers are from the code
  backgroundcolor=\color{white},  % choose the background color. You must add \usepackage{color}
  showspaces=false,               % show spaces adding particular underscores
  showstringspaces=false,         % underline spaces within strings
  showtabs=false,                 % show tabs within strings adding particular underscores
  tabsize=8,                      % sets default tabsize to 2 spaces
  captionpos=b,                   % sets the caption-position to bottom
  breaklines=true,                % sets automatic line breaking
  breakatwhitespace=true,         % sets if automatic breaks should only happen at whitespace
  title=\lstname,                 % show the filename of files included with \lstinputlisting;
}

%random comment

\newcommand{\cred}[1]{{\color{red}#1}}
\newcommand{\cblue}[1]{{\color{blue}#1}}

\newcommand{\toc}{\tableofcontents}


\def\name{Leon Leighton, Alec Merdler, Arthur Shing}

%% The following metadata will show up in the PDF properties
\hypersetup{
   colorlinks = true,
   urlcolor = black,
   linkcolor = black,
   pdfauthor = {\name},
   pdfkeywords = {cs444 ``operating systems'' files encryption block driver},
   pdftitle = {CS 444 Project 3: The Kernel Crypto API},
   pdfsubject = {CS 444 Project 3},
   pdfpagemode = UseNone
}

\parindent = 0.0 in
\parskip = 0.1 in


\begin{document}
\title{Project 3: The Kernel Crypto API}
\author{Leon Leighton, Alec Merdler, and Arthur Shing}

\begin{titlepage}
    \pagenumbering{gobble}
    \centering
    \maketitle
    \begin{abstract}
      This document provides an overview of the work done by Group 11-09 for Project 3: The Kernel Crypto API.
      It includes the design we used to implement the encryption algorithm.
      This includes answers to questions asked in the assignment description, a work log, and the git version control log.
    \end{abstract}


\end{titlepage}
\pagenumbering{arabic}
\tableofcontents
\clearpage

\section{Project 2}

\subsection{Design}
To begin, we based the structure of our design off of a simple block driver written by Jonathan Corbet and modified by Pat Patterson at http://blog.superpat.com/2010/05/04/a-simple-block-driver-for-linux-kernel-2-6-31, which was written for the 2.6.0 version of the kernel.

In our design, we use a 32 bit key for the encryption. Because this is a block cypher, we knew that we could have more data being read/written
than the amount of data that fits in one block. To accomodate for this, we encrypt/decrypt data in increments of block size.
We implemented the data encryption/decryption in the transfer function, where data is transferred to and from the device. 

 \\



\subsection{Answers}

% What do you think the main point of this assignment is?

We think that the main point of this assignment was to learn how to work with the Linux Kernel's Crypto API, and to practice working with memory allocation.
In this assignment, we created a RAM Disk device driver for the Linux Kernel and encrypted/decrypted data that is written and read by the device.

 \\

% How did you personally approach the problem? Design decisions, algorithm, etc.



 \\

% TESTING DETAILS

 \\

% What did you learn?

Overall, we learned how to implement block device drivers in the Linux Kernel and also how to work with the Kernel's Crypto API.
We also learned about the operations and methods that block devices have and use.
 \\


\subsection{Work Log}




\clearpage
\subsection{Version Control Log}



\FloatBarrier



\scalebox{0.8}{
\begin{tabular}{l l l l}\textbf{Detail} & \textbf{Author} & \textbf{Description} & \textbf{Date}\\\hline
\href{https://github.com/alecmerdler/cs-444-group-119/commit/25d90aae79f7ceb874554d67026829a300ab3d50}{25d90aa} & Arthur Shing & Added prelim git log & Mon May 8 20:09:42 2017 -0700\\\hline
\href{https://github.com/alecmerdler/cs-444-group-119/commit/456942fda5111035b42a118f86a8e811bda3ffdf}{456942f} & Arthur Shing & Merge branch 'master' of https://github.com/alecmerdler/cs-444-group-119 & Mon May 8 19:50:53 2017 -0700\\\hline
\href{https://github.com/alecmerdler/cs-444-group-119/commit/60dd43672b1c067427b0c275edec8d4ef9568fc4}{60dd436} & Arthur Shing & Added work log and answers to doc & Mon May 8 19:50:24 2017 -0700\\\hline
\href{https://github.com/alecmerdler/cs-444-group-119/commit/67f86bf465869f022751513f37527d58e15154e0}{67f86bf} & Leon Leighton & Switch to using safe version of list\_for\_each & Mon May 8 19:35:31 2017 -0700\\\hline
\href{https://github.com/alecmerdler/cs-444-group-119/commit/79df82650202a8951da4631a95190ab6efad0eb3}{79df826} & Leon Leighton & Add hw2 patch & Mon May 8 19:32:27 2017 -0700\\\hline
\href{https://github.com/alecmerdler/cs-444-group-119/commit/e30b48dcef3106f78ae673403bb5f29fae81ba88}{e30b48d} & Arthur Shing & Added comments on design problems & Mon May 8 19:29:02 2017 -0700\\\hline
\href{https://github.com/alecmerdler/cs-444-group-119/commit/5b4594c759d329266cf338240b7875263bc632d3}{5b4594c} & Arthur Shing & Added on to initial response for q2 & Mon May 8 18:14:48 2017 -0700\\\hline
\href{https://github.com/alecmerdler/cs-444-group-119/commit/059217a984ffbe3e3b44d015cbe592319b3dba6e}{059217a} & Arthur Shing & Added initial answers to q2 in doc for hw2 & Mon May 8 17:38:16 2017 -0700\\\hline
\href{https://github.com/alecmerdler/cs-444-group-119/commit/ed5a182d9e86faa8346f54f462fc370ef2b35c82}{ed5a182} & alecmerdler & better instructions & Mon May 8 13:03:15 2017 -0700\\\hline
\href{https://github.com/alecmerdler/cs-444-group-119/commit/8c395243ec5da5de07dcea98c75f0f8e2a0738ba}{8c39524} & alecmerdler & merged with master & Mon May 8 11:59:56 2017 -0700\\\hline
\href{https://github.com/alecmerdler/cs-444-group-119/commit/3faa0f9b30d0c483072dd175ee44914f4d44bc50}{3faa0f9} & alecmerdler & update readme & Mon May 8 11:58:44 2017 -0700\\\hline
\href{https://github.com/alecmerdler/cs-444-group-119/commit/5b27035b75b0a19f72ebf65f69aaf39e4248cea6}{5b27035} & Arthur Shing & Added answer to first question & Mon May 8 02:28:40 2017 -0700\\\hline
\href{https://github.com/alecmerdler/cs-444-group-119/commit/8adb95101e281849bf0a4302cc6cd14758bfbf56}{8adb951} & Arthur Shing & Cleaned up hw2 document for editing & Mon May 8 01:54:32 2017 -0700\\\hline
\href{https://github.com/alecmerdler/cs-444-group-119/commit/580dd714debbddba5d1bd94735ce59f269b24843}{580dd71} & Arthur Shing & Copied over old doc to hw 2 & Mon May 8 01:32:36 2017 -0700\\\hline
\href{https://github.com/alecmerdler/cs-444-group-119/commit/dcf43aea0a4a1c8e570394621cfb964ebc77ef02}{dcf43ae} & alecmerdler & added Kconfig.iosched and Makefile based on instructions & Sun May 7 14:58:28 2017 -0700\\\hline
\href{https://github.com/alecmerdler/cs-444-group-119/commit/b4c7044841e68579e335765d3370e40c4a786d8b}{b4c7044} & alecmerdler & improvements to readme & Sun May 7 14:08:29 2017 -0700\\\hline
\href{https://github.com/alecmerdler/cs-444-group-119/commit/3a4d07e9b71c259707f8763e93c01e7cddaf22b0}{3a4d07e} & alecmerdler & added TA instructions/tips & Sun May 7 14:01:04 2017 -0700\\\hline
\href{https://github.com/alecmerdler/cs-444-group-119/commit/ba155971591102e6b565e64634339b87b2c09268}{ba15597} & alecmerdler & more comments & Sun May 7 13:56:49 2017 -0700\\\hline
\href{https://github.com/alecmerdler/cs-444-group-119/commit/14edeb950fb33b0c392677ef4e68fec961602354}{14edeb9} & alecmerdler & adding documentation comments & Sun May 7 13:45:37 2017 -0700\\\hline
\href{https://github.com/alecmerdler/cs-444-group-119/commit/eb60ca356c9e4058a4477fcb159f83bf73349172}{eb60ca3} & alecmerdler & improving readme & Sun May 7 13:31:13 2017 -0700\\\hline
\href{https://github.com/alecmerdler/cs-444-group-119/commit/bb99ed237ceb79f1dd6b9ecfbb7447430c285fa4}{bb99ed2} & alecmerdler & working on I/O scheduler & Sun May 7 13:20:04 2017 -0700\\\hline
\href{https://github.com/alecmerdler/cs-444-group-119/commit/962a4d6b35cd24668972f637ae7bdd615edf0093}{962a4d6} & alecmerdler & actually deadlock is prevented because of blocking call to get left fork & Sat May 6 16:45:26 2017 -0700\\\hline
\href{https://github.com/alecmerdler/cs-444-group-119/commit/8f71c63a10bfc3765be8487bda3569ce52d3ad6a}{8f71c63} & alecmerdler & current implementation does not prevent deadlock & Sat May 6 16:12:52 2017 -0700\\\hline
\href{https://github.com/alecmerdler/cs-444-group-119/commit/9b37cb5b3b331734708d96a86f2a3ca0d2feaab9}{9b37cb5} & alecmerdler & improved docstrings & Sat May 6 14:15:44 2017 -0700\\\hline
\href{https://github.com/alecmerdler/cs-444-group-119/commit/9314c1fddf717564abcad3ecd2247886e962a64f}{9314c1f} & alecmerdler & closer to dining philosopher's solution & Sat May 6 14:11:51 2017 -0700\\\hline
\href{https://github.com/alecmerdler/cs-444-group-119/commit/79d98fc6b43d5d8c9363528696fc8da4b9afc90d}{79d98fc} & alecmerdler & cleanup & Sat May 6 12:28:35 2017 -0700\\\hline
\href{https://github.com/alecmerdler/cs-444-group-119/commit/44e8ccc0f6fd3c0f905660febea3a6162dac6327}{44e8ccc} & alecmerdler & starting scheduler & Sat May 6 11:53:50 2017 -0700\\\hline
\href{https://github.com/alecmerdler/cs-444-group-119/commit/b4123f84505625b4c2af77146950f44633652fda}{b4123f8} & alecmerdler & starting concurrency 2 & Sun Apr 30 19:11:51 2017 -0700\\\hline


\end{tabular}
}


\FloatBarrier
\end{document}
