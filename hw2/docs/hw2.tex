\documentclass[journal, letterpaper, draftclsnofoot, onecolumn, 10pt]{IEEEtran}

\usepackage{graphicx}
\usepackage{amssymb}
\usepackage{amsmath}
\usepackage{amsthm}

\usepackage{alltt}
\usepackage{float}
\usepackage{color}
\usepackage{url}
\usepackage{listings}

\usepackage{balance}
\usepackage[TABBOTCAP, tight]{subfigure}
\usepackage{enumitem}
\usepackage{pstricks, pst-node}
\usepackage{placeins}
\usepackage{geometry}
\usepackage{hyperref}
\geometry{textheight=8.5in, textwidth=6in}

\lstset{
  language=C,                % choose the language of the code
  numbers=left,                   % where to put the line-numbers
  stepnumber=1,                   % the step between two line-numbers.
  numbersep=5pt,                  % how far the line-numbers are from the code
  backgroundcolor=\color{white},  % choose the background color. You must add \usepackage{color}
  showspaces=false,               % show spaces adding particular underscores
  showstringspaces=false,         % underline spaces within strings
  showtabs=false,                 % show tabs within strings adding particular underscores
  tabsize=8,                      % sets default tabsize to 2 spaces
  captionpos=b,                   % sets the caption-position to bottom
  breaklines=true,                % sets automatic line breaking
  breakatwhitespace=true,         % sets if automatic breaks should only happen at whitespace
  title=\lstname,                 % show the filename of files included with \lstinputlisting;
}

%random comment

\newcommand{\cred}[1]{{\color{red}#1}}
\newcommand{\cblue}[1]{{\color{blue}#1}}

\newcommand{\toc}{\tableofcontents}


\def\name{Leon Leighton, Alec Merdler, Arthur Shing}

%% The following metadata will show up in the PDF properties
\hypersetup{
   colorlinks = true,
   urlcolor = black,
   linkcolor = black,
   pdfauthor = {\name},
   pdfkeywords = {cs444 ``operating systems'' files filesystem I/O},
   pdftitle = {CS 444 Project 2: I/O Elevators},
   pdfsubject = {CS 444 Project 2},
   pdfpagemode = UseNone
}

\parindent = 0.0 in
\parskip = 0.1 in


\begin{document}
\title{Project 1: Getting Acquainted}
\author{Leon Leighton, Alec Merdler, and Arthur Shing}

\begin{titlepage}
    \pagenumbering{gobble}
    \centering
    \maketitle
    \begin{abstract}
      This document provides an overview of the work done by Group 11-09 for Project 2: I/O Elevators.
      It includes the design we used to implement the SSTF algorithm.
      Also included is the write up for our work on the Concurrency 2 programming assignment.
      This includes answers to questions asked in the assignment description, a work log, and the git version control log.
    \end{abstract}


\end{titlepage}
\pagenumbering{arabic}
\tableofcontents
\clearpage

\section{Project 2}

\subsection{Design}
% Insertion sort



\subsection{Answers}

% What do you think the main point of this assignment is?

We think the main point of this assignment was to learn about I/O scheduling algorithms and to become familiar with implementing
changes to the linux kernel. In this exercise, we had to look into various files in the kernel to understand how to implement
the LOOK scheduler with the data structures provided in the kernel. Overall this assignment helped us in our understanding of
how the operating system schedules I/O requests.


% % How did you personally approach the problem? Design decisions, algorithm, etc.
%
% % HW 1 TEXT FOR REFERENCE
% We decided to utilize a mutex for the exercise to manage exclusive access to the buffer. A thread first locks the buffer, changes it,
% and then unlocks it. While the mutex is locked, other threads are blocked until it is unlocked. Threads are also blocked from the buffer
% while the buffer is full (for producers) or empty (for consumers). To implement the random number generator, we used a separate file that
% generated numbers using rdrand or the Mersenne Twister, depending on support.
%
% % How did you ensure your solution was correct? Testing details, for instance.
%
% % HW 1 TEXT FOR REFERENCE
% To test our implementation, we ran it with a smaller buffer to check if the blocks were occurring when the buffer was full or empty.
% We also tested random number generation and thread handling by running it with a large amount of threads.
%
% % What did you learn?
%
% Overall, we learned how to use pthreads in c, and how we should consider synchronization problems when it comes to multithreaded programming.
% We also learned how to correctly use version control systems, and the synchronization problems involved with those.

\subsection{Work Log}


\clearpage
\subsection{Version Control Log}



\FloatBarrier
\begin{table}[h!]
\centering
\caption{Git Log}
% \label{git-log}
\begin{tabular}{|c|c|c|c|c|c|c|}
    \hline \textbf{Version} & \textbf{Author} & \textbf{Date} & \textbf{commit message} & \textbf{MF} & \textbf{AL} & \textbf{DL}  \\
    \hline 1 & alecmerdler & 2017-04-11 & initial commit & 2 & 0 & 0 \\
    \hline 2 & alecmerdler & 2017-04-13 & added .gitignore & 1 & 1 & 0 \\
    \hline 3 & alecmerdler & 2017-04-14 & starting concurrency \#1 & 2 & 49 & 1 \\
    \hline 4 & Lee Leighton & 2017-04-18 & Add our\_rand code. Uses rdrand or mt19937 to return random number & 5 & 309 & 0 \\
    \hline 5 & Lee Leighton & 2017-04-20 & Docs folder with latex files. Initial write for the qemu portion. & 3 & 262 & 0 \\
    \hline 6 & Lee Leighton & 2017-04-20 & Merge pull request \#1 from leel8on/leel8on/random & 0 & 0 & 0 \\
    \hline 7 & Lee Leighton & 2017-04-21 & Changing where buffer is created. Add mutex locking and creation of item & 2 & 46 & 6 \\
    \hline 8 & Lee Leighton & 2017-04-21 & Fix how consumer access the item & 1 & 3 & 4 \\
    \hline 9 & Lee Leighton & 2017-04-21 & Add multiple threads for both consumer and producer specified as arguments & 1 & 28 & 16 \\
    \hline 10 & Lee Leighton & 2017-04-21 & Fix for loop index intialization & 1 & 3 & 3 \\
    \hline 11 & Lee Leighton & 2017-04-21 & Use same number of threads for both consumers and producers & 1 & 6 & 7 \\
    \hline 12 & Lee Leighton & 2017-04-21 & Keep track of threads ids in one array & 1 & 9 & 8 \\
    \hline 13 & Arthur Shing & 2017-04-21 & Moved seed generation to hw1.c & 2 & 6 & 3 \\
    \hline
\end{tabular}
\begin{tabular}{lp{12cm}}
  \label{tabular:legend:git-log}
  \textbf{acronym} & \textbf{meaning} \\
  V & \texttt{version} \\
  MF & Number of \texttt{modified files}. \\
  AL & Number of \texttt{added lines}. \\
  DL & Number of \texttt{deleted lines}. \\
\end{tabular}
\end{table}
\FloatBarrier
\end{document}
