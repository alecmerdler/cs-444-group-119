\documentclass[journal, letterpaper, draftclsnofoot, onecolumn, 10pt]{IEEEtran}

\usepackage{graphicx}
\usepackage{amssymb}
\usepackage{amsmath}
\usepackage{amsthm}

\usepackage{alltt}
\usepackage{float}
\usepackage{color}
\usepackage{url}
\usepackage{listings}

\usepackage{balance}
\usepackage[TABBOTCAP, tight]{subfigure}
\usepackage{enumitem}
\usepackage{pstricks, pst-node}
\usepackage{placeins}
\usepackage{geometry}
\usepackage{hyperref}
\geometry{textheight=8.5in, textwidth=6in}

\lstset{
  language=C,                % choose the language of the code
  numbers=left,                   % where to put the line-numbers
  stepnumber=1,                   % the step between two line-numbers.
  numbersep=5pt,                  % how far the line-numbers are from the code
  backgroundcolor=\color{white},  % choose the background color. You must add \usepackage{color}
  showspaces=false,               % show spaces adding particular underscores
  showstringspaces=false,         % underline spaces within strings
  showtabs=false,                 % show tabs within strings adding particular underscores
  tabsize=8,                      % sets default tabsize to 2 spaces
  captionpos=b,                   % sets the caption-position to bottom
  breaklines=true,                % sets automatic line breaking
  breakatwhitespace=true,         % sets if automatic breaks should only happen at whitespace
  title=\lstname,                 % show the filename of files included with \lstinputlisting;
}

%random comment

\newcommand{\cred}[1]{{\color{red}#1}}
\newcommand{\cblue}[1]{{\color{blue}#1}}

\newcommand{\toc}{\tableofcontents}


\def\name{Leon Leighton, Alec Merdler, Arthur Shing}

%% The following metadata will show up in the PDF properties
\hypersetup{
   colorlinks = true,
   urlcolor = black,
   linkcolor = black,
   pdfauthor = {\name},
   pdfkeywords = {cs444 ``operating systems'' files filesystem I/O},
   pdftitle = {CS 444 Project 2: I/O Elevators},
   pdfsubject = {CS 444 Project 2},
   pdfpagemode = UseNone
}

\parindent = 0.0 in
\parskip = 0.1 in


\begin{document}
\title{Project 1: Getting Acquainted}
\author{Leon Leighton, Alec Merdler, and Arthur Shing}

\begin{titlepage}
    \pagenumbering{gobble}
    \centering
    \maketitle
    \begin{abstract}
      This document provides an overview of the work done by Group 11-09 for Project 2: I/O Elevators.
      It includes the design we used to implement the SSTF algorithm.
      Also included is the write up for our work on the Concurrency 2 programming assignment.
      This includes answers to questions asked in the assignment description, a work log, and the git version control log.
    \end{abstract}


\end{titlepage}
\pagenumbering{arabic}
\tableofcontents
\clearpage

\section{Project 2}

\subsection{Design}
% Insertion sort



\subsection{Answers}

% What do you think the main point of this assignment is?

We think the main point of this assignment was to learn about I/O scheduling algorithms and to become familiar with implementing
changes to the linux kernel. In this exercise, we had to look into various files in the kernel to understand how to implement
the LOOK scheduler with the data structures provided in the kernel. Overall this assignment helped us in our understanding of
how the operating system schedules I/O requests. \\


% How did you personally approach the problem? Design decisions, algorithm, etc.

We decided to implement a C-LOOK I/O scheduler. To begin, we thought the simplest way would be to use the existing noop-iosched.c file
as our format and implement an insertion sort into the function where a request is added to the queue. This way, the queue would be
sorted and ready for dispatching. In addition, because the No-op is a FIFO implementation and does not account for read/write head
location, a shortest-seek-first implementation would require a way to include the last location and direction of the read/write head.
 \\

Our implementation thus far adds a sorting algorithm such that the requests are added in a sorted order. We decided not to implement
the sorting with respect to the location of the read/write head, meaning that the request queue is simply sorted in ascending order.
We saw that we had two choices: to sort the request queue in accordance with the read/write head location and dispatch the head of
the queue, or to sort it in ascending order and finding the correct request to dispatch according to the location and direction of
the read/write head. We decided to implement the latter because it overall seemed easier to sort and dispatch. \\

However, we have not implemented the dispatch function correctly. As a result, our current algorithm simply dispatches the request
at the head of the queue like noop.


% TESTING DETAILS
To test our implementation, we generated I/O on the VM through a python script that reads files from the directories starting at the
root directory ('/'). We then observed that the request were dispatched in queue in order of their sector numbers. We also used `dd'
create files on the disk and observed the same behavior. 



Overall, we learned about I/O scheduling algorithms and running scripts on the VM. We learned a lot about data structures
(specifically list\_head) in the kernel, as well as implementation of those data structures.\\

\subsection{Work Log}
We began work on 5/2. We revised our concurrency assignment 2 and began work on 5/4. We began by looking
at the existing code and determining what it was that we needed to change. Because of time conflicts, the three of us were often unable
to meet as a whole group, although we met in pairs. We cleaned up our implementation for the concurrency assignment on 5/6.
We created our implementation for the I/O scheduler on 5/7-8.

\clearpage
\subsection{Version Control Log}



\FloatBarrier
% \begin{table}[h!]
% \centering
% \caption{Git Log}

\scalebox{0.8}{
\begin{tabular}{l l l l}\textbf{Detail} & \textbf{Author} & \textbf{Description} & \textbf{Date}\\\hline
\href{https://github.com/alecmerdler/cs-444-group-119/commit/25d90aae79f7ceb874554d67026829a300ab3d50}{25d90aa} & Arthur Shing & Added prelim git log & Mon May 8 20:09:42 2017 -0700\\\hline
\href{https://github.com/alecmerdler/cs-444-group-119/commit/456942fda5111035b42a118f86a8e811bda3ffdf}{456942f} & Arthur Shing & Merge branch 'master' of https://github.com/alecmerdler/cs-444-group-119 & Mon May 8 19:50:53 2017 -0700\\\hline
\href{https://github.com/alecmerdler/cs-444-group-119/commit/60dd43672b1c067427b0c275edec8d4ef9568fc4}{60dd436} & Arthur Shing & Added work log and answers to doc & Mon May 8 19:50:24 2017 -0700\\\hline
\href{https://github.com/alecmerdler/cs-444-group-119/commit/67f86bf465869f022751513f37527d58e15154e0}{67f86bf} & Leon Leighton & Switch to using safe version of list\_for\_each & Mon May 8 19:35:31 2017 -0700\\\hline
\href{https://github.com/alecmerdler/cs-444-group-119/commit/79df82650202a8951da4631a95190ab6efad0eb3}{79df826} & Leon Leighton & Add hw2 patch & Mon May 8 19:32:27 2017 -0700\\\hline
\href{https://github.com/alecmerdler/cs-444-group-119/commit/e30b48dcef3106f78ae673403bb5f29fae81ba88}{e30b48d} & Arthur Shing & Added comments on design problems & Mon May 8 19:29:02 2017 -0700\\\hline
\href{https://github.com/alecmerdler/cs-444-group-119/commit/5b4594c759d329266cf338240b7875263bc632d3}{5b4594c} & Arthur Shing & Added on to initial response for q2 & Mon May 8 18:14:48 2017 -0700\\\hline
\href{https://github.com/alecmerdler/cs-444-group-119/commit/059217a984ffbe3e3b44d015cbe592319b3dba6e}{059217a} & Arthur Shing & Added initial answers to q2 in doc for hw2 & Mon May 8 17:38:16 2017 -0700\\\hline
\href{https://github.com/alecmerdler/cs-444-group-119/commit/ed5a182d9e86faa8346f54f462fc370ef2b35c82}{ed5a182} & alecmerdler & better instructions & Mon May 8 13:03:15 2017 -0700\\\hline
\href{https://github.com/alecmerdler/cs-444-group-119/commit/8c395243ec5da5de07dcea98c75f0f8e2a0738ba}{8c39524} & alecmerdler & merged with master & Mon May 8 11:59:56 2017 -0700\\\hline
\href{https://github.com/alecmerdler/cs-444-group-119/commit/3faa0f9b30d0c483072dd175ee44914f4d44bc50}{3faa0f9} & alecmerdler & update readme & Mon May 8 11:58:44 2017 -0700\\\hline
\href{https://github.com/alecmerdler/cs-444-group-119/commit/5b27035b75b0a19f72ebf65f69aaf39e4248cea6}{5b27035} & Arthur Shing & Added answer to first question & Mon May 8 02:28:40 2017 -0700\\\hline
\href{https://github.com/alecmerdler/cs-444-group-119/commit/8adb95101e281849bf0a4302cc6cd14758bfbf56}{8adb951} & Arthur Shing & Cleaned up hw2 document for editing & Mon May 8 01:54:32 2017 -0700\\\hline
\href{https://github.com/alecmerdler/cs-444-group-119/commit/580dd714debbddba5d1bd94735ce59f269b24843}{580dd71} & Arthur Shing & Copied over old doc to hw 2 & Mon May 8 01:32:36 2017 -0700\\\hline
\href{https://github.com/alecmerdler/cs-444-group-119/commit/dcf43aea0a4a1c8e570394621cfb964ebc77ef02}{dcf43ae} & alecmerdler & added Kconfig.iosched and Makefile based on instructions & Sun May 7 14:58:28 2017 -0700\\\hline
\href{https://github.com/alecmerdler/cs-444-group-119/commit/b4c7044841e68579e335765d3370e40c4a786d8b}{b4c7044} & alecmerdler & improvements to readme & Sun May 7 14:08:29 2017 -0700\\\hline
\href{https://github.com/alecmerdler/cs-444-group-119/commit/3a4d07e9b71c259707f8763e93c01e7cddaf22b0}{3a4d07e} & alecmerdler & added TA instructions/tips & Sun May 7 14:01:04 2017 -0700\\\hline
\href{https://github.com/alecmerdler/cs-444-group-119/commit/ba155971591102e6b565e64634339b87b2c09268}{ba15597} & alecmerdler & more comments & Sun May 7 13:56:49 2017 -0700\\\hline
\href{https://github.com/alecmerdler/cs-444-group-119/commit/14edeb950fb33b0c392677ef4e68fec961602354}{14edeb9} & alecmerdler & adding documentation comments & Sun May 7 13:45:37 2017 -0700\\\hline
\href{https://github.com/alecmerdler/cs-444-group-119/commit/eb60ca356c9e4058a4477fcb159f83bf73349172}{eb60ca3} & alecmerdler & improving readme & Sun May 7 13:31:13 2017 -0700\\\hline
\href{https://github.com/alecmerdler/cs-444-group-119/commit/bb99ed237ceb79f1dd6b9ecfbb7447430c285fa4}{bb99ed2} & alecmerdler & working on I/O scheduler & Sun May 7 13:20:04 2017 -0700\\\hline
\href{https://github.com/alecmerdler/cs-444-group-119/commit/962a4d6b35cd24668972f637ae7bdd615edf0093}{962a4d6} & alecmerdler & actually deadlock is prevented because of blocking call to get left fork & Sat May 6 16:45:26 2017 -0700\\\hline
\href{https://github.com/alecmerdler/cs-444-group-119/commit/8f71c63a10bfc3765be8487bda3569ce52d3ad6a}{8f71c63} & alecmerdler & current implementation does not prevent deadlock & Sat May 6 16:12:52 2017 -0700\\\hline
\href{https://github.com/alecmerdler/cs-444-group-119/commit/9b37cb5b3b331734708d96a86f2a3ca0d2feaab9}{9b37cb5} & alecmerdler & improved docstrings & Sat May 6 14:15:44 2017 -0700\\\hline
\href{https://github.com/alecmerdler/cs-444-group-119/commit/9314c1fddf717564abcad3ecd2247886e962a64f}{9314c1f} & alecmerdler & closer to dining philosopher's solution & Sat May 6 14:11:51 2017 -0700\\\hline
\href{https://github.com/alecmerdler/cs-444-group-119/commit/79d98fc6b43d5d8c9363528696fc8da4b9afc90d}{79d98fc} & alecmerdler & cleanup & Sat May 6 12:28:35 2017 -0700\\\hline
\href{https://github.com/alecmerdler/cs-444-group-119/commit/44e8ccc0f6fd3c0f905660febea3a6162dac6327}{44e8ccc} & alecmerdler & starting scheduler & Sat May 6 11:53:50 2017 -0700\\\hline
\href{https://github.com/alecmerdler/cs-444-group-119/commit/b4123f84505625b4c2af77146950f44633652fda}{b4123f8} & alecmerdler & starting concurrency 2 & Sun Apr 30 19:11:51 2017 -0700\\\hline


\end{tabular}
}
% % \label{git-log}
% \begin{tabular}{|c|c|c|c|}
%     \hline \textbf{Version} & \textbf{Author} & \textbf{Date} & \textbf{commit message}  \\
%     \hline 1 & alecmerdler & 2017-04-30 & starting concurrency 2 \\
%     \hline 1 & alecmerdler & 2017-05-06 & starting scheduler \\
%     \hline 1 & alecmerdler & 2017-05-06 & cleanup \\
%     \hline 1 & alecmerdler & 2017-05-06 & closer to dining philosopher's solution \\
%     \hline 1 & alecmerdler & 2017-05-06 & improved docstrings\\
%     \hline 1 & alecmerdler & 2017-05-06 & current implementation does not prevent deadlock \\
%     \hline 1 & alecmerdler & 2017-05-06 & actually deadlock is prevented because of blocking call to get left fork \\
%     \hline 1 & alecmerdler & 2017-05-07 & working on I/O scheduler \\
%     \hline 1 & alecmerdler & 2017-05-07 & improving readme \\
%     \hline 1 & alecmerdler & 2017-05-07 & adding documentation comments \\
%     \hline 1 & alecmerdler & 2017-05-07 & more comments \\
%     \hline 1 & alecmerdler & 2017-05-07 & added TA instructions/tips \\
%     \hline 1 & alecmerdler & 2017-05-07 & improvements to readme\\
%     \hline 1 & alecmerdler & 2017-05-07 & added Kconfig.iosched and Makefile based on instructions \\
%     \hline 1 & Arthur Shing & 2017-05-08 & Cleaned up hw2 doc for editing\\
%     \hline 1 & alecmerdler & 2017-05-08 & update readme\\
%     \hline 1 & alecmerdler & 2017-05-08 & merged with master\\
%     \hline 1 & alecmerdler & 2017-05-08 & better instructions\\
%     \hline 1 & Arthur Shing & 2017-05-08 & Added initial responses \\
%     \hline 1 & Arthur Shing & 2017-05-08 & Added comments on design problems \\
%     \hline 1 & Lee Leighton & 2017-05-08 & Added hw2 patch \\
%     \hline 1 & Lee Leighton & 2017-05-08 & Switch to using safe version of list\_for\_each \\
%     \hline
% \end{tabular}
% \end{table}
\FloatBarrier
\end{document}
